\chapter{Spiking Neural Networks: Models and Learning Algorithms}

\section{Multi-agent cooperation over cellular network}

\section{Neuron model}

The neuron model in this paper is the Leaky Integrate-and-Fire (LIF). The LIF model is a biological model that can be represented as a circuit with a resistor and capacitor and represents a first-order dynamic system \cite{Ch5_NM1},

\begin{equation} \label{Eq.1}
    R_{m}C_{m}\frac{dV_m\left(t\right)}{dt} = E_{l} - V_m\left(t\right) + R_{m}I\left(t\right)
\end{equation}

\noindent where $V_{m}(t)$ is the neuron's membrane potential, $R_{m}$ is the membrane resistance, $C_{m}$ is the membrane capacitance, $E_{l}$ is the resting potential, and $I(t)$ is the input current. The neuron spikes when its potential reaches the threshold potential ($V_{th}$). The potential of the neuron immediately reaches the reset potential ($V_{res}$) after it spikes.

The spike rate is a parameter that determines how fast the neuron spikes \cite{Ch5_NM2}.

\begin{equation} \label{Eq.4}
    r_{[Hz]} = \frac{1}{t_{isi}\enspace [s]}
\end{equation}

\noindent where $t_{isi}$ is the inter-spike interval that can be calculated using the neuron model. When the potential of a neuron reaches the threshold potential, it fires. Therefore, based on the analytical solution of (\ref{Eq.1}), the inter-spike interval time can be written as,

\begin{equation} \label{Eq.7}
    t_{isi} = \tau_{m}\ln\left(\frac{E_{l} + R_{m}I - V_{res}}{E_{l} + R_{m}I - V_{th}}\right)
\end{equation}

\noindent where $\tau_{m}$ is the membrane time constant. 

According to (\ref{Eq.7}), the following condition should be satisfied to have a finite value for $t_{isi}$,

\begin{equation} \label{Eq.8}
    E_{l} + R_{m}I - V_{th} > 0
\end{equation}

\noindent or

\begin{equation} \label{Eq.9}
    I > \frac{V_{th} - E_{l}}{R_{m}}
\end{equation}

\noindent which means that the input current higher than the above value generates spikes. 

After calculating the minimum input for neurons, we must find the maximum input based on the inter-spike interval. Equation (\ref{Eq.7}) can be written as,

\begin{equation} \label{Eq.10}
    t_{isi} = \tau_{m}\ln\left(1+\frac{V_{th} - V_{res}}{E_{l} + R_{m}I - V_{th}}\right)
\end{equation}

\noindent Equation (\ref{Eq.10}) can be approximated using the Maclaurin series for the natural logarithm function ($\ln(1+z)\approx z$) as follows,

\begin{equation} \label{Eq.11}
    t_{isi} = \frac{\tau_{m}\left(V_{th} - V_{res}\right)}{E_{l} + R_{m}I - V_{th}}
\end{equation}

\noindent Solving for $I$, an input current as a function of the inter-spike interval can be obtained,

\begin{equation} \label{Eq.12}
    I = \frac{\tau_{m}\left(V_{th} - V_{res}\right)}{t_{isi}R_{m}} + \frac{V_{th} - E_{l}}{R_{m}}
\end{equation}

The maximum value for the input current makes the neuron fire at each sample time ($\Delta t$). Therefore, the maximum input current is,

\begin{equation} \label{Eq.13}
    I^{max} = \frac{\tau_{m}\left(V_{th} - V_{res}\right)}{\Delta t R_{m}} + \frac{V_{th} - E_{l}}{R_{m}}
\end{equation}

In this section, we obtained the minimum and maximum values for input current using (\ref{Eq.9}) and (\ref{Eq.13}). These equations are used in the learning and encoding processes of the SNN.
