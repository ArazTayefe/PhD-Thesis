\chapter{Discussion and Conclusion}
\label{chap:discussion_and_conclusion}

In this thesis, we have explored the integration of Spiking Neural Networks (SNNs) with multiagent reinforcement learning in decentralized edge computing environments. The research has demonstrated the potential of biologically inspired neural models to enhance learning efficiency, robustness, and adaptability in dynamic multiagent systems. The key contributions of this work can be summarized as follows:
\begin{itemize}
    \item Development of adaptive mechanisms for SNNs that stabilize learning under sparse reward conditions, balancing synaptic plasticity with network equilibrium.
    \item Implementation of federated learning strategies to facilitate lightweight policy sharing among agents, reducing communication overhead while preserving decentralized autonomy.
    \item Design and simulation of a vision-based multiagent docking framework utilizing Dynamic Vision Sensor (DVS) event streams and deep SNN controllers, achieving successful cooperative docking of a substantial payload within stringent performance criteria.
    \item Comprehensive evaluation of the proposed systems, demonstrating their effectiveness in real-world cooperative tasks and their resilience to operational constraints.
  \end{itemize}
The results obtained from the simulations indicate that SNNs can significantly enhance the performance of multiagent systems in edge computing scenarios. The adaptive learning mechanisms developed in this thesis have shown promise in addressing the challenges posed by sparse and delayed rewards, which are common in real-world applications.
Furthermore, the developed approach has proven effective in enabling agents to share knowledge without the need for centralized coordination, thus maintaining the benefits of decentralization. The vision-based docking framework has successfully demonstrated the practical application of SNNs in complex cooperative tasks, showcasing their potential for real-time decision-making and control.


While this research has made significant strides in the integration of SNNs with multiagent reinforcement learning, several avenues for future work remain. Future research could focus on the following areas:
\begin{itemize}
    \item \textbf{Neuron Model Diversity:} Investigating the implications of neuron model diversity within SNNs, examining how heterogeneity affects learning outcomes and system performance.
    \item \textbf{Advanced Optimization Techniques:} Employing methodologies such as Neuroevolution of Augmenting Topologies (NEAT), transfer learning, and evolutionary optimization algorithms to optimize the global model.
    \item \textbf{Real-World Deployment:} Deveolping the \ac{fl} methods to nonhomogeneous multiagent systems and testing the proposed frameworks in real-world scenarios to validate their effectiveness and robustness.
\end{itemize}
By addressing these areas, future research can further enhance the capabilities of multiagent systems, utilizing the synergies between SNNs and federated learning in dynamic and potentially adversarial environments. This work aims to improve the efficiency of bio-inspired neural network models in edge computing and ensure the integrity and privacy of data in decentralized networks.
