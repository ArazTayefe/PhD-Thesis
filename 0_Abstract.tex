\chapter*{Abstract}

This proposal outlines an innovative approach to multiagent reinforcement learning by integrating Spiking Neural Networks (SNNs) with Federated Learning (FL), aimed at advancing the capabilities of edge computing. This integration promises notable improvements in learning efficiency and data privacy but presents unique challenges requiring further investigation. The research will focus on four pivotal areas: enhancing learning stability in SNNs, reducing communication overhead in FL, exploring the heterogeneity of spiking neuron models, and strengthening resilience against model poisoning attacks.

This research aims to develop adaptive mechanisms for SNNs to maintain learning stability amidst dynamic environmental rewards, thereby addressing the critical balance between synaptic plasticity and equilibrium. The proposal also aims to innovate in managing communication overhead by introducing adaptive communication strategies that ensure efficiency without compromising responsiveness, leveraging algorithms that reduce redundancy in model updates.

Furthermore, the research will investigate the implications of neuron model diversity within SNNs, examining how heterogeneity affects learning outcomes and system performance. This includes employing methodologies such as Neuroevolution of Augmenting Topologies (NEAT), transfer learning, and evolutionary optimization algorithms to optimize the global model. Lastly, the proposal emphasizes enhancing the system's defense against model poisoning attacks, proposing the use of robust statistical methods to detect and mitigate such risks.

By addressing these areas, this research aims to significantly contribute to the development of more resilient, efficient, and adaptive multiagent systems, leveraging the synergies between SNNs and FL in dynamic and potentially adversarial environments. This work aims to improve the efficiency of bio-inspired neural network models in edge computing and ensure the integrity and privacy of data in decentralized networks.